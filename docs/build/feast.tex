%% Generated by Sphinx.
\def\sphinxdocclass{report}
\documentclass[letterpaper,10pt,english]{sphinxmanual}
\ifdefined\pdfpxdimen
   \let\sphinxpxdimen\pdfpxdimen\else\newdimen\sphinxpxdimen
\fi \sphinxpxdimen=.75bp\relax

\PassOptionsToPackage{warn}{textcomp}
\usepackage[utf8]{inputenc}
\ifdefined\DeclareUnicodeCharacter
% support both utf8 and utf8x syntaxes
  \ifdefined\DeclareUnicodeCharacterAsOptional
    \def\sphinxDUC#1{\DeclareUnicodeCharacter{"#1}}
  \else
    \let\sphinxDUC\DeclareUnicodeCharacter
  \fi
  \sphinxDUC{00A0}{\nobreakspace}
  \sphinxDUC{2500}{\sphinxunichar{2500}}
  \sphinxDUC{2502}{\sphinxunichar{2502}}
  \sphinxDUC{2514}{\sphinxunichar{2514}}
  \sphinxDUC{251C}{\sphinxunichar{251C}}
  \sphinxDUC{2572}{\textbackslash}
\fi
\usepackage{cmap}
\usepackage[T1]{fontenc}
\usepackage{amsmath,amssymb,amstext}
\usepackage{babel}



\usepackage{times}
\expandafter\ifx\csname T@LGR\endcsname\relax
\else
% LGR was declared as font encoding
  \substitutefont{LGR}{\rmdefault}{cmr}
  \substitutefont{LGR}{\sfdefault}{cmss}
  \substitutefont{LGR}{\ttdefault}{cmtt}
\fi
\expandafter\ifx\csname T@X2\endcsname\relax
  \expandafter\ifx\csname T@T2A\endcsname\relax
  \else
  % T2A was declared as font encoding
    \substitutefont{T2A}{\rmdefault}{cmr}
    \substitutefont{T2A}{\sfdefault}{cmss}
    \substitutefont{T2A}{\ttdefault}{cmtt}
  \fi
\else
% X2 was declared as font encoding
  \substitutefont{X2}{\rmdefault}{cmr}
  \substitutefont{X2}{\sfdefault}{cmss}
  \substitutefont{X2}{\ttdefault}{cmtt}
\fi


\usepackage[Bjarne]{fncychap}
\usepackage{sphinx}

\fvset{fontsize=\small}
\usepackage{geometry}


% Include hyperref last.
\usepackage{hyperref}
% Fix anchor placement for figures with captions.
\usepackage{hypcap}% it must be loaded after hyperref.
% Set up styles of URL: it should be placed after hyperref.
\urlstyle{same}
\addto\captionsenglish{\renewcommand{\contentsname}{Contents:}}

\usepackage{sphinxmessages}
\setcounter{tocdepth}{4}
\setcounter{secnumdepth}{4}


\title{FEAST}
\date{Nov 30, 2020}
\release{3.1}
\author{Chandler Kemp and Clay Bell}
\newcommand{\sphinxlogo}{\vbox{}}
\renewcommand{\releasename}{Release}
\makeindex
\begin{document}

\pagestyle{empty}
\sphinxmaketitle
\pagestyle{plain}
\sphinxtableofcontents
\pagestyle{normal}
\phantomsection\label{\detokenize{index::doc}}

\begin{quote}
\end{quote}


\chapter{FEAST modules}
\label{\detokenize{index:feast-modules}}

\section{DetectionModules}
\label{\detokenize{index:detectionmodules}}

\subsection{abstract\_detection\_method}
\label{\detokenize{index:module-feast.DetectionModules.abstract_detection_method}}\label{\detokenize{index:abstract-detection-method}}\index{feast.DetectionModules.abstract\_detection\_method (module)@\spxentry{feast.DetectionModules.abstract\_detection\_method}\spxextra{module}}
This module contains an abstract class that all DetectionMethods should inherit.


\subsubsection{DetectionMethod}
\label{\detokenize{index:detectionmethod}}\index{DetectionMethod (class in feast.DetectionModules.abstract\_detection\_method)@\spxentry{DetectionMethod}\spxextra{class in feast.DetectionModules.abstract\_detection\_method}}

\begin{fulllineitems}
\phantomsection\label{\detokenize{index:feast.DetectionModules.abstract_detection_method.DetectionMethod}}\pysiglinewithargsret{\sphinxbfcode{\sphinxupquote{class }}\sphinxbfcode{\sphinxupquote{DetectionMethod}}}{\emph{time}, \emph{detection\_variables=None}, \emph{op\_envelope=None}, \emph{ophrs=None}}{}
DetectionMethod is an abstract super class that defines the form required for all detection methods
\begin{quote}\begin{description}
\item[{Parameters}] \leavevmode\begin{itemize}
\item {} 
\sphinxstyleliteralstrong{\sphinxupquote{time}} \textendash{} a Time object

\item {} 
\sphinxstyleliteralstrong{\sphinxupquote{detection\_variables}} \textendash{} list of variable names to be used in the detection calculations (eg. {[}‘wind speed’{]})

\item {} 
\sphinxstyleliteralstrong{\sphinxupquote{op\_envelope}} \textendash{} operating envelope specifications for the detection method

\item {} 
\sphinxstyleliteralstrong{\sphinxupquote{ophrs}} \textendash{} a dict specifying operating hours for the DetectionMethod

\end{itemize}

\end{description}\end{quote}

\end{fulllineitems}



\subsection{comp\_survey}
\label{\detokenize{index:module-feast.DetectionModules.comp_survey}}\label{\detokenize{index:comp-survey}}\index{feast.DetectionModules.comp\_survey (module)@\spxentry{feast.DetectionModules.comp\_survey}\spxextra{module}}
This module defines the component level survey based detection class, CompSurvey.


\subsubsection{CompSurvey}
\label{\detokenize{index:compsurvey}}\index{CompSurvey (class in feast.DetectionModules.comp\_survey)@\spxentry{CompSurvey}\spxextra{class in feast.DetectionModules.comp\_survey}}

\begin{fulllineitems}
\phantomsection\label{\detokenize{index:feast.DetectionModules.comp_survey.CompSurvey}}\pysiglinewithargsret{\sphinxbfcode{\sphinxupquote{class }}\sphinxbfcode{\sphinxupquote{CompSurvey}}}{\emph{time}, \emph{dispatch\_object}, \emph{survey\_interval}, \emph{survey\_speed}, \emph{labor}, \emph{site\_queue}, \emph{detection\_probability\_points}, \emph{detection\_probabilities}, \emph{ophrs}, \emph{comp\_survey\_index=0}, \emph{site\_survey\_index=0}, \emph{op\_env\_wait\_time=7}, \emph{**kwargs}}{}
This class specifies a component level, survey based detection method.
A component level method identifies the specific component that is the source of the
emission at the time of detection. Examples of components include connectors, valves, open ended lines, etc.
A survey based method inspects emissions at specific moments in time (as opposed to a monitor method that
continuously monitors for emissions).
The class has three essential attributes:
\begin{enumerate}
\sphinxsetlistlabels{\arabic}{enumi}{enumii}{}{.}%
\item {} 
An operating envelope function to determine if conditions satisfy requirements for the method to be deployed

\item {} 
A probability of detection surface function to determine which emissions are detected

\item {} 
The ability to call a follow up action

\end{enumerate}
\begin{quote}\begin{description}
\item[{Parameters}] \leavevmode\begin{itemize}
\item {} 
\sphinxstyleliteralstrong{\sphinxupquote{time}} \textendash{} a Time object

\item {} 
\sphinxstyleliteralstrong{\sphinxupquote{dispatch\_object}} \textendash{} an object to dispatch for follow\sphinxhyphen{}up actions (typically a Repair method)

\item {} 
\sphinxstyleliteralstrong{\sphinxupquote{survey\_interval}} \textendash{} Time between surveys (float\textendash{}days)

\item {} 
\sphinxstyleliteralstrong{\sphinxupquote{survey\_speed}} \textendash{} Speed of surveys (float\textendash{}components/hr)

\item {} 
\sphinxstyleliteralstrong{\sphinxupquote{labor}} \textendash{} Cost of surveys (float\textendash{}\$/hr)

\item {} 
\sphinxstyleliteralstrong{\sphinxupquote{site\_queue}} \textendash{} Sites to survey (list of ints)

\item {} 
\sphinxstyleliteralstrong{\sphinxupquote{detection\_probability\_points}} \textendash{} Set of conditions at which the probability was measured (array)

\item {} 
\sphinxstyleliteralstrong{\sphinxupquote{detection\_probabilities}} \textendash{} Set of probabilities that were measured (array)

\item {} 
\sphinxstyleliteralstrong{\sphinxupquote{comp\_survey\_index}} \textendash{} Index of the component to be surveyed next (int)

\item {} 
\sphinxstyleliteralstrong{\sphinxupquote{site\_survey\_index}} \textendash{} Index of the site to be surveyed next (or currently under survey) (int)

\item {} 
\sphinxstyleliteralstrong{\sphinxupquote{op\_env\_wait\_time}} \textendash{} Time to wait if operating envelope conditions fail part way through a site before
moving to the next site (float\textendash{}days)

\item {} 
\sphinxstyleliteralstrong{\sphinxupquote{ophrs}} \textendash{} range of times of day when the method performs survey (dict\textendash{}hours). eg: \{‘begin’: 8, ‘end’: 17\}

\end{itemize}

\end{description}\end{quote}

\end{fulllineitems}



\subsection{ldar\_program}
\label{\detokenize{index:module-feast.DetectionModules.ldar_program}}\label{\detokenize{index:ldar-program}}\index{feast.DetectionModules.ldar\_program (module)@\spxentry{feast.DetectionModules.ldar\_program}\spxextra{module}}
This module defines the LDARProgram class.


\subsubsection{LDARProgram}
\label{\detokenize{index:ldarprogram}}\index{LDARProgram (class in feast.DetectionModules.ldar\_program)@\spxentry{LDARProgram}\spxextra{class in feast.DetectionModules.ldar\_program}}

\begin{fulllineitems}
\phantomsection\label{\detokenize{index:feast.DetectionModules.ldar_program.LDARProgram}}\pysiglinewithargsret{\sphinxbfcode{\sphinxupquote{class }}\sphinxbfcode{\sphinxupquote{LDARProgram}}}{\emph{time}, \emph{gas\_field}, \emph{tech\_dict}}{}
An LDAR program contains one or more detection methods and one or more repair methods. Each LDAR program records
the find and repair costs associated with all detection and repair methods in the program. The LDAR program
deploys runs the action methods of each detection and repair method contained in the program. The detection and
repair methods determine their own behavior at each time step.
\begin{quote}\begin{description}
\item[{Parameters}] \leavevmode\begin{itemize}
\item {} 
\sphinxstyleliteralstrong{\sphinxupquote{time}} \textendash{} a Time object

\item {} 
\sphinxstyleliteralstrong{\sphinxupquote{gas\_field}} \textendash{} a GasField object

\item {} 
\sphinxstyleliteralstrong{\sphinxupquote{tech\_dict}} \textendash{} a dict containing all of the detection methods to be employed by the LDAR program. The dict
must have the form \{“name”: DetectionMethod\}. All of the relationships between detection methods and between
detection methods and repair methods must be defined by the dispatch\_objects specified for each method.

\end{itemize}

\end{description}\end{quote}

\end{fulllineitems}



\subsection{repair}
\label{\detokenize{index:module-feast.DetectionModules.repair}}\label{\detokenize{index:repair}}\index{feast.DetectionModules.repair (module)@\spxentry{feast.DetectionModules.repair}\spxextra{module}}
This module defines the Repair class. Repair may be called by detection objects as follow up actions.


\subsubsection{Repair}
\label{\detokenize{index:id1}}\index{Repair (class in feast.DetectionModules.repair)@\spxentry{Repair}\spxextra{class in feast.DetectionModules.repair}}

\begin{fulllineitems}
\phantomsection\label{\detokenize{index:feast.DetectionModules.repair.Repair}}\pysiglinewithargsret{\sphinxbfcode{\sphinxupquote{class }}\sphinxbfcode{\sphinxupquote{Repair}}}{\emph{repair\_delay=0}, \emph{name=None}}{}
Defines a repair process. A repair process determines when emissions are ended by an LDAR program and the
associated costs.
\begin{quote}\begin{description}
\item[{Parameters}] \leavevmode
\sphinxstyleliteralstrong{\sphinxupquote{repair\_delay}} \textendash{} The time between when an emission is passed to Repair and when it is removed from the
simulation (float\textendash{}days)

\end{description}\end{quote}

\end{fulllineitems}



\subsection{site\_monitor}
\label{\detokenize{index:module-feast.DetectionModules.site_monitor}}\label{\detokenize{index:site-monitor}}\index{feast.DetectionModules.site\_monitor (module)@\spxentry{feast.DetectionModules.site\_monitor}\spxextra{module}}
The site\_monitor module defines the site monitor detection class, SiteMonitor.


\subsubsection{SiteMonitor}
\label{\detokenize{index:sitemonitor}}\index{SiteMonitor (class in feast.DetectionModules.site\_monitor)@\spxentry{SiteMonitor}\spxextra{class in feast.DetectionModules.site\_monitor}}

\begin{fulllineitems}
\phantomsection\label{\detokenize{index:feast.DetectionModules.site_monitor.SiteMonitor}}\pysiglinewithargsret{\sphinxbfcode{\sphinxupquote{class }}\sphinxbfcode{\sphinxupquote{SiteMonitor}}}{\emph{time}, \emph{dispatch\_object}, \emph{time\_to\_detect\_points}, \emph{time\_to\_detect\_days}, \emph{ophrs=None}, \emph{capital=0}, \emph{site\_queue=None}, \emph{**kwargs}}{}
This class specifies a site level continuous monitoring method. A site monitor continuously observes
emissions from an entire site and determines when an action should be dispatched at the site.
The method has three essential characteristics:
\begin{enumerate}
\sphinxsetlistlabels{\arabic}{enumi}{enumii}{}{.}%
\item {} 
A list of the sites where the method applies

\item {} 
A time\sphinxhyphen{}to\sphinxhyphen{}detect surface specified as a list of conditions and associated mean detection times

\item {} 
The ability to dispatch a follow up action

\end{enumerate}
\begin{quote}\begin{description}
\item[{Parameters}] \leavevmode\begin{itemize}
\item {} 
\sphinxstyleliteralstrong{\sphinxupquote{time}} \textendash{} a Time object

\item {} 
\sphinxstyleliteralstrong{\sphinxupquote{dispatch\_object}} \textendash{} the object to dispatch for follow up actions

\item {} 
\sphinxstyleliteralstrong{\sphinxupquote{time\_to\_detect\_points}} \textendash{} The conditions at which the time to detection was measured. (NxM array,
where N is the number of distinct conditions and M is the number of variables (up to two)).

\item {} 
\sphinxstyleliteralstrong{\sphinxupquote{time\_to\_detect\_days}} \textendash{} The list of probabilities of detection associated with every point in
detection\_probability\_points (array of shape N, where N is the number of conditions with an associated
probability of detection).

\item {} 
\sphinxstyleliteralstrong{\sphinxupquote{ophrs}} \textendash{} 
The times of day when the SiteMonitor is operational. Should be a dict:

\{‘begin’: hour integer, ‘end’: hour integer\}


\item {} 
\sphinxstyleliteralstrong{\sphinxupquote{capital}} \textendash{} The total cost of installing the site monitor system in the simulation (float\textendash{}\$)

\item {} 
\sphinxstyleliteralstrong{\sphinxupquote{site\_queue}} \textendash{} A list of sites where the site monitor system is installed

\end{itemize}

\end{description}\end{quote}

\end{fulllineitems}



\subsection{site\_survey}
\label{\detokenize{index:module-feast.DetectionModules.site_survey}}\label{\detokenize{index:site-survey}}\index{feast.DetectionModules.site\_survey (module)@\spxentry{feast.DetectionModules.site\_survey}\spxextra{module}}
The site\_survey module defines the site level level survey based detection class, SiteSurvey.


\subsubsection{SiteSurvey}
\label{\detokenize{index:sitesurvey}}\index{SiteSurvey (class in feast.DetectionModules.site\_survey)@\spxentry{SiteSurvey}\spxextra{class in feast.DetectionModules.site\_survey}}

\begin{fulllineitems}
\phantomsection\label{\detokenize{index:feast.DetectionModules.site_survey.SiteSurvey}}\pysiglinewithargsret{\sphinxbfcode{\sphinxupquote{class }}\sphinxbfcode{\sphinxupquote{SiteSurvey}}}{\emph{time}, \emph{dispatch\_object}, \emph{sites\_per\_day}, \emph{site\_cost}, \emph{detection\_probability\_points}, \emph{detection\_probabilities}, \emph{op\_envelope=None}, \emph{ophrs=None}, \emph{site\_queue=None}, \emph{survey\_interval=None}, \emph{**kwargs}}{}
SiteSurvey specifies a site level, survey based detection method. A site level detection method is sensitive to
the total emissions from a site. If emissions are detected, the site is identified as the source of emissions
rather than a component on the site. Survey based detection methods search for emissions at a specific moment in
time (as opposed to monitor detection methods that continuously scan sites for new emissions).
The class has three essential attributes:
\begin{enumerate}
\sphinxsetlistlabels{\arabic}{enumi}{enumii}{}{.}%
\item {} 
An operating envelope function to determine if the detection method can be applied

\item {} 
A probability of detection surface function to determine which emissions are detected

\item {} 
The ability to dispatch a follow up action

\end{enumerate}
\begin{quote}\begin{description}
\item[{Parameters}] \leavevmode\begin{itemize}
\item {} 
\sphinxstyleliteralstrong{\sphinxupquote{time}} \textendash{} a Time object

\item {} 
\sphinxstyleliteralstrong{\sphinxupquote{dispatch\_object}} \textendash{} the object that SiteSurvey will pass flagged site indexes to (DetectionMethod or Repair)

\item {} 
\sphinxstyleliteralstrong{\sphinxupquote{sites\_per\_day}} \textendash{} the number of sites that the method can survey in one day (int)

\item {} 
\sphinxstyleliteralstrong{\sphinxupquote{site\_cost}} \textendash{} the cost per site of the detection method (\$/site\textendash{}float)

\item {} 
\sphinxstyleliteralstrong{\sphinxupquote{detection\_probability\_points}} \textendash{} The conditions at which the detection probability was measured. (NxM
array, where N is the number of distinct conditions and M is the number of variables (up to two)).

\item {} 
\sphinxstyleliteralstrong{\sphinxupquote{detection\_probabilities}} \textendash{} The list of probabilities of detection associated with every point in
detection\_probability\_points (array of shape N, where N is the number of conditions with an associated
probability of detection).

\item {} 
\sphinxstyleliteralstrong{\sphinxupquote{op\_envelope}} \textendash{} 
The set of conditions underwhich the SiteSurvey may operate. The op\_envelope must be
passed as a dict with the following form\textendash{}

\{‘parameter name’: \{‘class’: int, ‘min’: list of minimum conditions, ‘max’: list of maximum conditions\}\}

Unique minima can be defined for every site in a list if the op\_envelope ‘class’ is site specific. Multiple
minima can be defined in a list for a single site if multiple ranges should be considered.


\item {} 
\sphinxstyleliteralstrong{\sphinxupquote{ophrs}} \textendash{} 
The times of day when the SiteSurvey can be deployed. Should be a dict:

\{‘begin’: hour integer, ‘end’: hour integer\}


\item {} 
\sphinxstyleliteralstrong{\sphinxupquote{site\_queue}} \textendash{} an ordered list of sites to be surveyed. An LDAR program may update this list.

\item {} 
\sphinxstyleliteralstrong{\sphinxupquote{survey\_interval}} \textendash{} The time between surveys (int\textendash{}days)

\end{itemize}

\end{description}\end{quote}

\end{fulllineitems}



\section{EmissionSimModules}
\label{\detokenize{index:emissionsimmodules}}

\subsection{emission\_class\_functions}
\label{\detokenize{index:module-feast.EmissionSimModules.emission_class_functions}}\label{\detokenize{index:emission-class-functions}}\index{feast.EmissionSimModules.emission\_class\_functions (module)@\spxentry{feast.EmissionSimModules.emission\_class\_functions}\spxextra{module}}
A class for storing emission properties and functions for modifying emission proporeties throughout a simulation are
defined in this module.


\subsubsection{Emission}
\label{\detokenize{index:emission}}\index{Emission (class in feast.EmissionSimModules.emission\_class\_functions)@\spxentry{Emission}\spxextra{class in feast.EmissionSimModules.emission\_class\_functions}}

\begin{fulllineitems}
\phantomsection\label{\detokenize{index:feast.EmissionSimModules.emission_class_functions.Emission}}\pysiglinewithargsret{\sphinxbfcode{\sphinxupquote{class }}\sphinxbfcode{\sphinxupquote{Emission}}}{\emph{flux=()}, \emph{reparable=True}, \emph{site\_index=()}, \emph{comp\_index=()}, \emph{start\_time=0}, \emph{end\_time=inf}, \emph{repair\_cost=()}, \emph{emission\_id=None}}{}
Stores all properties of all emissions that exist at a particular instant in a simulation.
\begin{quote}\begin{description}
\item[{Parameters}] \leavevmode\begin{itemize}
\item {} 
\sphinxstyleliteralstrong{\sphinxupquote{flux}} \textendash{} An array of emission rates (array of floats\textendash{}gram/second)

\item {} 
\sphinxstyleliteralstrong{\sphinxupquote{reparable}} \textendash{} An array of True/False values to indicate whether or not an emission is reparable

\item {} 
\sphinxstyleliteralstrong{\sphinxupquote{site\_index}} \textendash{} An array indicating the index of the site that contains every emission

\item {} 
\sphinxstyleliteralstrong{\sphinxupquote{comp\_index}} \textendash{} An array indicating the index of the component that is the source of each emission

\item {} 
\sphinxstyleliteralstrong{\sphinxupquote{start\_time}} \textendash{} An array specifying the time when every emission begins

\item {} 
\sphinxstyleliteralstrong{\sphinxupquote{end\_time}} \textendash{} An array specifying the time when every emission will end (days)

\item {} 
\sphinxstyleliteralstrong{\sphinxupquote{emission\_id}} \textendash{} 

\item {} 
\sphinxstyleliteralstrong{\sphinxupquote{repair\_cost}} \textendash{} An array storing the cost of repairing every emission (\$)

\end{itemize}

\end{description}\end{quote}

\end{fulllineitems}



\subsubsection{bootstrap\_emission\_maker}
\label{\detokenize{index:bootstrap-emission-maker}}\index{bootstrap\_emission\_maker() (in module feast.EmissionSimModules.emission\_class\_functions)@\spxentry{bootstrap\_emission\_maker()}\spxextra{in module feast.EmissionSimModules.emission\_class\_functions}}

\begin{fulllineitems}
\phantomsection\label{\detokenize{index:feast.EmissionSimModules.emission_class_functions.bootstrap_emission_maker}}\pysiglinewithargsret{\sphinxbfcode{\sphinxupquote{bootstrap\_emission\_maker}}}{\emph{n\_em\_in}, \emph{comp\_name}, \emph{site}, \emph{time}, \emph{start\_time=None}, \emph{reparable=True}}{}
Create leaks using a bootstrap method.
\begin{quote}\begin{description}
\item[{Parameters}] \leavevmode\begin{itemize}
\item {} 
\sphinxstyleliteralstrong{\sphinxupquote{n\_em\_in}} \textendash{} number of leaks to generate

\item {} 
\sphinxstyleliteralstrong{\sphinxupquote{comp\_name}} \textendash{} key to a Component object in site.comp\_dict

\item {} 
\sphinxstyleliteralstrong{\sphinxupquote{site}} \textendash{} a Site object

\item {} 
\sphinxstyleliteralstrong{\sphinxupquote{time}} \textendash{} a Time object

\item {} 
\sphinxstyleliteralstrong{\sphinxupquote{start\_time}} \textendash{} the times at which each emission begins

\item {} 
\sphinxstyleliteralstrong{\sphinxupquote{reparable}} \textendash{} Specifies whether emissions should be reparable or not (boolean)

\end{itemize}

\end{description}\end{quote}

\end{fulllineitems}



\subsubsection{comp\_indexes\_fcn}
\label{\detokenize{index:comp-indexes-fcn}}\index{comp\_indexes\_fcn() (in module feast.EmissionSimModules.emission\_class\_functions)@\spxentry{comp\_indexes\_fcn()}\spxextra{in module feast.EmissionSimModules.emission\_class\_functions}}

\begin{fulllineitems}
\phantomsection\label{\detokenize{index:feast.EmissionSimModules.emission_class_functions.comp_indexes_fcn}}\pysiglinewithargsret{\sphinxbfcode{\sphinxupquote{comp\_indexes\_fcn}}}{\emph{site}, \emph{comp\_name}, \emph{n\_inds}}{}
Returns an array of indexes to associate with new emissions
\begin{quote}\begin{description}
\item[{Parameters}] \leavevmode\begin{itemize}
\item {} 
\sphinxstyleliteralstrong{\sphinxupquote{site}} \textendash{} a EmissionSimModules.simulation\_classes.Site object

\item {} 
\sphinxstyleliteralstrong{\sphinxupquote{comp\_name}} \textendash{} name of a component contained in Site.comp\_dict

\item {} 
\sphinxstyleliteralstrong{\sphinxupquote{n\_inds}} \textendash{} Integer of indexes to generate

\end{itemize}

\item[{Returns}] \leavevmode
An array of indexes in the range specified for the relevant component

\end{description}\end{quote}

\end{fulllineitems}



\subsubsection{emission\_objects\_generator}
\label{\detokenize{index:emission-objects-generator}}\index{emission\_objects\_generator() (in module feast.EmissionSimModules.emission\_class\_functions)@\spxentry{emission\_objects\_generator()}\spxextra{in module feast.EmissionSimModules.emission\_class\_functions}}

\begin{fulllineitems}
\phantomsection\label{\detokenize{index:feast.EmissionSimModules.emission_class_functions.emission_objects_generator}}\pysiglinewithargsret{\sphinxbfcode{\sphinxupquote{emission\_objects\_generator}}}{\emph{dist\_type}, \emph{emission\_data\_path}, \emph{custom\_emission\_maker=None}}{}
emission\_objects\_generator is a parent function that will be called to initialize gas fields
\begin{quote}\begin{description}
\item[{Parameters}] \leavevmode\begin{itemize}
\item {} 
\sphinxstyleliteralstrong{\sphinxupquote{dist\_type}} \textendash{} Type of leak distribution to be used

\item {} 
\sphinxstyleliteralstrong{\sphinxupquote{leak\_data\_path}} \textendash{} Path to a leak data file

\end{itemize}

\end{description}\end{quote}

\end{fulllineitems}



\subsubsection{permitted\_emission}
\label{\detokenize{index:permitted-emission}}\index{permitted\_emission() (in module feast.EmissionSimModules.emission\_class\_functions)@\spxentry{permitted\_emission()}\spxextra{in module feast.EmissionSimModules.emission\_class\_functions}}

\begin{fulllineitems}
\phantomsection\label{\detokenize{index:feast.EmissionSimModules.emission_class_functions.permitted_emission}}\pysiglinewithargsret{\sphinxbfcode{\sphinxupquote{permitted\_emission}}}{\emph{n\_emit}, \emph{sizes}, \emph{duration}, \emph{time}, \emph{site}, \emph{comp\_name}, \emph{start\_time}}{}
Creates an emission object specifying new permitted emissions
\begin{quote}\begin{description}
\item[{Parameters}] \leavevmode\begin{itemize}
\item {} 
\sphinxstyleliteralstrong{\sphinxupquote{n\_emit}} \textendash{} number of emissions to create

\item {} 
\sphinxstyleliteralstrong{\sphinxupquote{sizes}} \textendash{} a list of leak sizes from which to specify the emission rate

\item {} 
\sphinxstyleliteralstrong{\sphinxupquote{duration}} \textendash{} a float defining the duration of the emission

\item {} 
\sphinxstyleliteralstrong{\sphinxupquote{time}} \textendash{} a Time object

\item {} 
\sphinxstyleliteralstrong{\sphinxupquote{site}} \textendash{} a Site object

\item {} 
\sphinxstyleliteralstrong{\sphinxupquote{comp\_name}} \textendash{} Name of the component to be considered from within site.comp\_dict

\item {} 
\sphinxstyleliteralstrong{\sphinxupquote{start\_times}} \textendash{} array of times at which emissions start

\end{itemize}

\item[{Returns}] \leavevmode
an Emission object

\end{description}\end{quote}

\end{fulllineitems}



\subsection{infrastructure\_classes}
\label{\detokenize{index:module-feast.EmissionSimModules.infrastructure_classes}}\label{\detokenize{index:infrastructure-classes}}\index{feast.EmissionSimModules.infrastructure\_classes (module)@\spxentry{feast.EmissionSimModules.infrastructure\_classes}\spxextra{module}}
This module stores component, gasfield and site classes to represent infrastructure in a simulation


\subsubsection{Component}
\label{\detokenize{index:component}}\index{Component (class in feast.EmissionSimModules.infrastructure\_classes)@\spxentry{Component}\spxextra{class in feast.EmissionSimModules.infrastructure\_classes}}

\begin{fulllineitems}
\phantomsection\label{\detokenize{index:feast.EmissionSimModules.infrastructure_classes.Component}}\pysiglinewithargsret{\sphinxbfcode{\sphinxupquote{class }}\sphinxbfcode{\sphinxupquote{Component}}}{\emph{repair\_cost\_path=None, emission\_data\_path=None, base\_reparable=None, custom\_emission\_maker=None, emission\_production\_rate=0, emission\_per\_comp=None, episodic\_emission\_sizes={[}0{]}, episodic\_emission\_per\_day=0, episodic\_emission\_duration=0, vent\_sizes={[}0{]}, vent\_period=inf, vent\_starts=array({[}{]}, dtype=float64), vent\_duration=0, name=\textquotesingle{}default\textquotesingle{}, null\_repair\_rate=None, dist\_type=\textquotesingle{}bootstrap\textquotesingle{}}}{}
A class to store parameters defining a component (for example, name, leak production rate, leak size
distribution, etc)
\begin{quote}\begin{description}
\item[{Parameters}] \leavevmode\begin{itemize}
\item {} 
\sphinxstyleliteralstrong{\sphinxupquote{repair\_cost\_path}} \textendash{} path to a repair cost data file

\item {} 
\sphinxstyleliteralstrong{\sphinxupquote{emission\_data\_path}} \textendash{} path to an emission data file

\item {} 
\sphinxstyleliteralstrong{\sphinxupquote{base\_reparable}} \textendash{} Defines whether emissions generated are reparable with a boolean true/false

\item {} 
\sphinxstyleliteralstrong{\sphinxupquote{custom\_emission\_maker}} \textendash{} Optional custom defined function for creating new emissions

\item {} 
\sphinxstyleliteralstrong{\sphinxupquote{emission\_production\_rate}} \textendash{} The rate at which new emissions are created (emissions per day per component)

\item {} 
\sphinxstyleliteralstrong{\sphinxupquote{emission\_per\_comp}} \textendash{} The number of emissions expected per component (must be less than 1)
If emission\_per\_comp is left as None, then emission\_per\_comp is set equal to the emissions per component
recorded in the file at emission\_data\_path.

\item {} 
\sphinxstyleliteralstrong{\sphinxupquote{episodic\_emission\_sizes}} \textendash{} A list of emission sizes to draw from for episodic emissions (g/s)

\item {} 
\sphinxstyleliteralstrong{\sphinxupquote{episodic\_emission\_per\_day}} \textendash{} The average frequency at which episodic emissions occur (1/days)

\item {} 
\sphinxstyleliteralstrong{\sphinxupquote{episodic\_emission\_duration}} \textendash{} The duration of episodic emissions (days)

\item {} 
\sphinxstyleliteralstrong{\sphinxupquote{vent\_sizes}} \textendash{} A list of emission sizes for periodic emissions (g/s)

\item {} 
\sphinxstyleliteralstrong{\sphinxupquote{vent\_period}} \textendash{} The time between emissions (days)

\item {} 
\sphinxstyleliteralstrong{\sphinxupquote{vent\_duration}} \textendash{} the time that a periodic vent persits (days)

\item {} 
\sphinxstyleliteralstrong{\sphinxupquote{vent\_starts}} \textendash{} the time at which the first periodic vent occurs at each component in the simulation

\item {} 
\sphinxstyleliteralstrong{\sphinxupquote{name}} \textendash{} A name for the instance of Component

\item {} 
\sphinxstyleliteralstrong{\sphinxupquote{null\_repair\_rate}} \textendash{} the rate at which fugitive emissions are repaired. If None, a steady state
assumption is enforced based on emission\_production\_rate and emission\_per\_comp.

\item {} 
\sphinxstyleliteralstrong{\sphinxupquote{dist\_type}} \textendash{} The type of distribution to be used in determining emission rates for new emissions

\end{itemize}

\end{description}\end{quote}

\end{fulllineitems}



\subsubsection{GasField}
\label{\detokenize{index:gasfield}}\index{GasField (class in feast.EmissionSimModules.infrastructure\_classes)@\spxentry{GasField}\spxextra{class in feast.EmissionSimModules.infrastructure\_classes}}

\begin{fulllineitems}
\phantomsection\label{\detokenize{index:feast.EmissionSimModules.infrastructure_classes.GasField}}\pysiglinewithargsret{\sphinxbfcode{\sphinxupquote{class }}\sphinxbfcode{\sphinxupquote{GasField}}}{\emph{time=None}, \emph{sites=None}, \emph{emissions=None}, \emph{met\_data\_path=None}}{}
GasField accommodates all data that defines a gas field at the beginning of a simulation.
\begin{quote}\begin{description}
\item[{Parameters}] \leavevmode\begin{itemize}
\item {} 
\sphinxstyleliteralstrong{\sphinxupquote{time}} \textendash{} A FEAST time object

\item {} 
\sphinxstyleliteralstrong{\sphinxupquote{sites}} \textendash{} a dict of sites like this: \{‘name’: \{‘number’: n\_sites, ‘parameters’: site\_object\}\}

\item {} 
\sphinxstyleliteralstrong{\sphinxupquote{emissions}} \textendash{} A FEAST emission object to be used during the simulations

\item {} 
\sphinxstyleliteralstrong{\sphinxupquote{met\_data\_path}} \textendash{} A path to a met data file

\end{itemize}

\end{description}\end{quote}

\end{fulllineitems}



\subsubsection{Site}
\label{\detokenize{index:site}}\index{Site (class in feast.EmissionSimModules.infrastructure\_classes)@\spxentry{Site}\spxextra{class in feast.EmissionSimModules.infrastructure\_classes}}

\begin{fulllineitems}
\phantomsection\label{\detokenize{index:feast.EmissionSimModules.infrastructure_classes.Site}}\pysiglinewithargsret{\sphinxbfcode{\sphinxupquote{class }}\sphinxbfcode{\sphinxupquote{Site}}}{\emph{name=\textquotesingle{}default\textquotesingle{}}, \emph{comp\_dict=None}, \emph{prod\_dat=None}}{}
A class to store the number and type of components associated with a site.
\begin{quote}\begin{description}
\item[{Parameters}] \leavevmode\begin{itemize}
\item {} 
\sphinxstyleliteralstrong{\sphinxupquote{name}} \textendash{} The name of the site object (a string)

\item {} 
\sphinxstyleliteralstrong{\sphinxupquote{comp\_dict}} \textendash{} A dict of components at the site, for example:
\{‘name’: \{‘number’: 650, ‘parameters’: Component()\}\}

\item {} 
\sphinxstyleliteralstrong{\sphinxupquote{prod\_dat}} \textendash{} 

\end{itemize}

\end{description}\end{quote}

\end{fulllineitems}



\subsection{result\_classes}
\label{\detokenize{index:module-feast.EmissionSimModules.result_classes}}\label{\detokenize{index:result-classes}}\index{feast.EmissionSimModules.result\_classes (module)@\spxentry{feast.EmissionSimModules.result\_classes}\spxextra{module}}
result\_classes defines classes that are used to store event counts and continuous variable data for saving.


\subsubsection{ResultAggregate}
\label{\detokenize{index:resultaggregate}}\index{ResultAggregate (class in feast.EmissionSimModules.result\_classes)@\spxentry{ResultAggregate}\spxextra{class in feast.EmissionSimModules.result\_classes}}

\begin{fulllineitems}
\phantomsection\label{\detokenize{index:feast.EmissionSimModules.result_classes.ResultAggregate}}\pysiglinewithargsret{\sphinxbfcode{\sphinxupquote{class }}\sphinxbfcode{\sphinxupquote{ResultAggregate}}}{\emph{units=None}, \emph{time\_value=None}}{}
A super class designed to store aggregate results during a simulation.
Time and value pairs are stored in a list.

\end{fulllineitems}



\subsubsection{ResultContinuous}
\label{\detokenize{index:resultcontinuous}}\index{ResultContinuous (class in feast.EmissionSimModules.result\_classes)@\spxentry{ResultContinuous}\spxextra{class in feast.EmissionSimModules.result\_classes}}

\begin{fulllineitems}
\phantomsection\label{\detokenize{index:feast.EmissionSimModules.result_classes.ResultContinuous}}\pysiglinewithargsret{\sphinxbfcode{\sphinxupquote{class }}\sphinxbfcode{\sphinxupquote{ResultContinuous}}}{\emph{**kwargs}}{}
Designed to store continuous rates that endure between consecutive time recordings as opposed to discrete
variables that occur at a specific time. For example, emission rate can be recorded as a continuous data type.

\end{fulllineitems}



\subsubsection{ResultDiscrete}
\label{\detokenize{index:resultdiscrete}}\index{ResultDiscrete (class in feast.EmissionSimModules.result\_classes)@\spxentry{ResultDiscrete}\spxextra{class in feast.EmissionSimModules.result\_classes}}

\begin{fulllineitems}
\phantomsection\label{\detokenize{index:feast.EmissionSimModules.result_classes.ResultDiscrete}}\pysiglinewithargsret{\sphinxbfcode{\sphinxupquote{class }}\sphinxbfcode{\sphinxupquote{ResultDiscrete}}}{\emph{**kwargs}}{}
Designed to store discrete values associated with specific times, as opposed to continuous rates that persist
between consecutive data points. For example, the number of sites surveyed can be recorded as a discrete data type.

\end{fulllineitems}



\subsection{simulation\_classes}
\label{\detokenize{index:module-feast.EmissionSimModules.simulation_classes}}\label{\detokenize{index:simulation-classes}}\index{feast.EmissionSimModules.simulation\_classes (module)@\spxentry{feast.EmissionSimModules.simulation\_classes}\spxextra{module}}
simulation\_classes stores the classes used to represent time, results and financial settings in simulations.


\subsubsection{Scenario}
\label{\detokenize{index:scenario}}\index{Scenario (class in feast.EmissionSimModules.simulation\_classes)@\spxentry{Scenario}\spxextra{class in feast.EmissionSimModules.simulation\_classes}}

\begin{fulllineitems}
\phantomsection\label{\detokenize{index:feast.EmissionSimModules.simulation_classes.Scenario}}\pysiglinewithargsret{\sphinxbfcode{\sphinxupquote{class }}\sphinxbfcode{\sphinxupquote{Scenario}}}{\emph{time}, \emph{gas\_field}, \emph{ldar\_program\_dict}}{}
A class to store all data specifying a scenario and the methods to run and save a realization
\begin{quote}\begin{description}
\item[{Parameters}] \leavevmode\begin{itemize}
\item {} 
\sphinxstyleliteralstrong{\sphinxupquote{time}} \textendash{} Time object

\item {} 
\sphinxstyleliteralstrong{\sphinxupquote{gas\_field}} \textendash{} GasField object

\item {} 
\sphinxstyleliteralstrong{\sphinxupquote{ldar\_program\_dict}} \textendash{} dict of detection methods and associated data

\end{itemize}

\end{description}\end{quote}

\end{fulllineitems}



\subsubsection{Time}
\label{\detokenize{index:time}}\index{Time (class in feast.EmissionSimModules.simulation\_classes)@\spxentry{Time}\spxextra{class in feast.EmissionSimModules.simulation\_classes}}

\begin{fulllineitems}
\phantomsection\label{\detokenize{index:feast.EmissionSimModules.simulation_classes.Time}}\pysiglinewithargsret{\sphinxbfcode{\sphinxupquote{class }}\sphinxbfcode{\sphinxupquote{Time}}}{\emph{delta\_t=1}, \emph{end\_time=365}, \emph{current\_time=0}}{}
Instances of the time class store all time related information during a simulation
\begin{quote}\begin{description}
\item[{Parameters}] \leavevmode\begin{itemize}
\item {} 
\sphinxstyleliteralstrong{\sphinxupquote{delta\_t}} \textendash{} length of one timestep (days)

\item {} 
\sphinxstyleliteralstrong{\sphinxupquote{end\_time}} \textendash{} length of the simulation (days)

\item {} 
\sphinxstyleliteralstrong{\sphinxupquote{current\_time}} \textendash{} current time in a simulation (days)

\end{itemize}

\end{description}\end{quote}

\end{fulllineitems}



\section{input\_data\_classes}
\label{\detokenize{index:module-feast.input_data_classes}}\label{\detokenize{index:input-data-classes}}\index{feast.input\_data\_classes (module)@\spxentry{feast.input\_data\_classes}\spxextra{module}}
This module defines all classes used to store input data.


\subsection{DataFile}
\label{\detokenize{index:datafile}}\index{DataFile (class in feast.input\_data\_classes)@\spxentry{DataFile}\spxextra{class in feast.input\_data\_classes}}

\begin{fulllineitems}
\phantomsection\label{\detokenize{index:feast.input_data_classes.DataFile}}\pysiglinewithargsret{\sphinxbfcode{\sphinxupquote{class }}\sphinxbfcode{\sphinxupquote{DataFile}}}{\emph{notes=\textquotesingle{}No notes provided\textquotesingle{}}, \emph{raw\_file\_name=None}, \emph{data\_prep\_file=None}}{}
DataFile is an abstract super class that all data file types inherit from FEAST.
\begin{quote}\begin{description}
\item[{Parameters}] \leavevmode\begin{itemize}
\item {} 
\sphinxstyleliteralstrong{\sphinxupquote{notes}} \textendash{} A string containing notes on the object created

\item {} 
\sphinxstyleliteralstrong{\sphinxupquote{raw\_file\_name}} \textendash{} path (or list of paths) to a raw input file(s)

\item {} 
\sphinxstyleliteralstrong{\sphinxupquote{data\_prep\_file}} \textendash{} path to the file used to process input data and create the DataFile object

\end{itemize}

\end{description}\end{quote}

\end{fulllineitems}



\subsection{LeakData}
\label{\detokenize{index:leakdata}}\index{LeakData (class in feast.input\_data\_classes)@\spxentry{LeakData}\spxextra{class in feast.input\_data\_classes}}

\begin{fulllineitems}
\phantomsection\label{\detokenize{index:feast.input_data_classes.LeakData}}\pysiglinewithargsret{\sphinxbfcode{\sphinxupquote{class }}\sphinxbfcode{\sphinxupquote{LeakData}}}{\emph{notes=\textquotesingle{}No notes provided\textquotesingle{}}, \emph{raw\_file\_name=None}, \emph{data\_prep\_file=None}, \emph{leak\_sizes=None}}{}
LeakData is designed to store all leak size data from a reference. It accommodates multiple detection methods
within a single instance.

Creates a LeakData object
\begin{quote}\begin{description}
\item[{Parameters}] \leavevmode\begin{itemize}
\item {} 
\sphinxstyleliteralstrong{\sphinxupquote{notes}} \textendash{} a string containing notes on the object created

\item {} 
\sphinxstyleliteralstrong{\sphinxupquote{raw\_file\_name}} \textendash{} path to a raw input file

\item {} 
\sphinxstyleliteralstrong{\sphinxupquote{data\_prep\_file}} \textendash{} path to a script used to build the object from the raw data file

\item {} 
\sphinxstyleliteralstrong{\sphinxupquote{leak\_sizes}} \textendash{} list of leak sizes. If leaks were detected using multiple methods, leak\_sizes must be a dict
with one key for each detection method

\end{itemize}

\end{description}\end{quote}

\end{fulllineitems}



\subsection{ProductionData}
\label{\detokenize{index:productiondata}}\index{ProductionData (class in feast.input\_data\_classes)@\spxentry{ProductionData}\spxextra{class in feast.input\_data\_classes}}

\begin{fulllineitems}
\phantomsection\label{\detokenize{index:feast.input_data_classes.ProductionData}}\pysiglinewithargsret{\sphinxbfcode{\sphinxupquote{class }}\sphinxbfcode{\sphinxupquote{ProductionData}}}{\emph{site\_prod=None}, \emph{**kwargs}}{}
Stores an array of production rates that may be associated with gas production sites
\begin{quote}\begin{description}
\item[{Parameters}] \leavevmode\begin{itemize}
\item {} 
\sphinxstyleliteralstrong{\sphinxupquote{prod\_dat}} \textendash{} an array of production rates

\item {} 
\sphinxstyleliteralstrong{\sphinxupquote{kwargs}} \textendash{} pass through

\end{itemize}

\end{description}\end{quote}

\end{fulllineitems}



\subsection{RepairData}
\label{\detokenize{index:repairdata}}\index{RepairData (class in feast.input\_data\_classes)@\spxentry{RepairData}\spxextra{class in feast.input\_data\_classes}}

\begin{fulllineitems}
\phantomsection\label{\detokenize{index:feast.input_data_classes.RepairData}}\pysiglinewithargsret{\sphinxbfcode{\sphinxupquote{class }}\sphinxbfcode{\sphinxupquote{RepairData}}}{\emph{notes=\textquotesingle{}No notes provided\textquotesingle{}}, \emph{raw\_file\_name=None}}{}
RepairData is designed to store the costs of repairing leaks from a particular reference andd associated notes.
\begin{quote}\begin{description}
\item[{Parameters}] \leavevmode\begin{itemize}
\item {} 
\sphinxstyleliteralstrong{\sphinxupquote{notes}} \textendash{} A string containing notes on the object created

\item {} 
\sphinxstyleliteralstrong{\sphinxupquote{raw\_file\_name}} \textendash{} path to a raw input file

\end{itemize}

\end{description}\end{quote}

\end{fulllineitems}



\section{MEET\_1\_importer}
\label{\detokenize{index:module-feast.MEET_1_importer}}\label{\detokenize{index:meet-1-importer}}\index{feast.MEET\_1\_importer (module)@\spxentry{feast.MEET\_1\_importer}\spxextra{module}}

\subsection{gas\_comp\_to\_dict}
\label{\detokenize{index:gas-comp-to-dict}}\index{gas\_comp\_to\_dict() (in module feast.MEET\_1\_importer)@\spxentry{gas\_comp\_to\_dict()}\spxextra{in module feast.MEET\_1\_importer}}

\begin{fulllineitems}
\phantomsection\label{\detokenize{index:feast.MEET_1_importer.gas_comp_to_dict}}\pysiglinewithargsret{\sphinxbfcode{\sphinxupquote{gas\_comp\_to\_dict}}}{\emph{gc}, \emph{delta\_t}}{}
converts a gas\_comp DataFrame to a dict specifying the start time, top time, emission rate, location,
emission type and emission ID for every emitter in gas\_comp. If an emission changes its emission rate,
it is recorded as stopping and restarting at the time of the change. LDAR programs will treat the emission
correctly due to the emission ID that persists after the change in emission rate.
\begin{quote}\begin{description}
\item[{Parameters}] \leavevmode\begin{itemize}
\item {} 
\sphinxstyleliteralstrong{\sphinxupquote{gc}} \textendash{} a gas\_comp DataFrame as created by the function load\_gas\_comp\_file

\item {} 
\sphinxstyleliteralstrong{\sphinxupquote{delta\_t}} \textendash{} The time resolution of the gas\_comp file (seconds)

\end{itemize}

\item[{Returns}] \leavevmode
the dict containing emitter data from the gas\_comp file

\end{description}\end{quote}

\end{fulllineitems}



\subsection{gascomp\_reader}
\label{\detokenize{index:gascomp-reader}}\index{gascomp\_reader() (in module feast.MEET\_1\_importer)@\spxentry{gascomp\_reader()}\spxextra{in module feast.MEET\_1\_importer}}

\begin{fulllineitems}
\phantomsection\label{\detokenize{index:feast.MEET_1_importer.gascomp_reader}}\pysiglinewithargsret{\sphinxbfcode{\sphinxupquote{gascomp\_reader}}}{\emph{path\_to\_gas\_comp}, \emph{feast\_delta\_t}, \emph{duration}, \emph{comps\_per\_site}, \emph{rep\_cost\_path}, \emph{met\_data\_path=None}, \emph{met\_start\_hr=None}}{}
Read a MEET 1.0 GasComp result file and return a feast Time object and GasField object specifying all of the
emissions to occur in a FEAST simulation
\begin{quote}\begin{description}
\item[{Parameters}] \leavevmode\begin{itemize}
\item {} 
\sphinxstyleliteralstrong{\sphinxupquote{path\_to\_gas\_comp}} \textendash{} path to a gas\_comp file

\item {} 
\sphinxstyleliteralstrong{\sphinxupquote{feast\_delta\_t}} \textendash{} time resolution of the FEAST simulation (days). FEAST will represent emissions with the     precision of the original MEET simulation, but will invoke LDAR events as though they occur instantaneously with     the time resolution specified here.

\item {} 
\sphinxstyleliteralstrong{\sphinxupquote{duration}} \textendash{} duration of the simulation to run in FEAST (days)

\item {} 
\sphinxstyleliteralstrong{\sphinxupquote{comps\_per\_site}} \textendash{} A dict of form \{loc: number of components\}. This will specify the number of components to     be inspected for leaks at every site. Must have a key for every location ID in the gas\_comp file.

\item {} 
\sphinxstyleliteralstrong{\sphinxupquote{rep\_cost\_path}} \textendash{} Path to a repair cost data file. The FEAST will randomly assign these repair costs to leaks     simulated in MEET.

\item {} 
\sphinxstyleliteralstrong{\sphinxupquote{met\_data\_path}} \textendash{} Path to a meteorological data file (if left as None, FEAST will run without met data)

\item {} 
\sphinxstyleliteralstrong{\sphinxupquote{met\_start\_hr}} \textendash{} FEAST will rotate the meteorological data to begin at the hour specified by an integer here.

\end{itemize}

\item[{Return time\_obj}] \leavevmode
a FEAST Time object to specify a simulation

\item[{Return gas\_field}] \leavevmode
a FEAST GasField object

\end{description}\end{quote}

\end{fulllineitems}



\subsection{gc\_dat\_to\_gas\_field}
\label{\detokenize{index:gc-dat-to-gas-field}}\index{gc\_dat\_to\_gas\_field() (in module feast.MEET\_1\_importer)@\spxentry{gc\_dat\_to\_gas\_field()}\spxextra{in module feast.MEET\_1\_importer}}

\begin{fulllineitems}
\phantomsection\label{\detokenize{index:feast.MEET_1_importer.gc_dat_to_gas_field}}\pysiglinewithargsret{\sphinxbfcode{\sphinxupquote{gc\_dat\_to\_gas\_field}}}{\emph{feast\_delta\_t}, \emph{duration}, \emph{comps\_per\_site}, \emph{rep\_cost\_path}, \emph{met\_data\_path=None}, \emph{met\_start\_hr=None}, \emph{size=0}, \emph{loc=0}, \emph{start=0}, \emph{stop=0}, \emph{emtype=0}, \emph{em\_id=None}}{}
Ports data from a data\_dict returned by the funcation gas\_comp\_to\_dict into a FEAST GasField
\begin{quote}\begin{description}
\item[{Parameters}] \leavevmode\begin{itemize}
\item {} 
\sphinxstyleliteralstrong{\sphinxupquote{feast\_delta\_t}} \textendash{} FEAST time resolution (days)

\item {} 
\sphinxstyleliteralstrong{\sphinxupquote{duration}} \textendash{} FEAST simulation duration (days)

\item {} 
\sphinxstyleliteralstrong{\sphinxupquote{comps\_per\_site}} \textendash{} A dict of form \{loc: number of components\}. This will specify the number of components to     be inspected for leaks at every site

\item {} 
\sphinxstyleliteralstrong{\sphinxupquote{rep\_cost\_path}} \textendash{} Path to a repair cost data file. The FEAST will randomly assign these repair costs to leaks     simulated in MEET

\item {} 
\sphinxstyleliteralstrong{\sphinxupquote{met\_data\_path}} \textendash{} Path to a meteorological data file (if left as None, FEAST will run without met data)

\item {} 
\sphinxstyleliteralstrong{\sphinxupquote{met\_start\_hr}} \textendash{} FEAST will rotate the meteorological data to begin at the hour specified by an integer here.

\item {} 
\sphinxstyleliteralstrong{\sphinxupquote{size}} \textendash{} an array of emission sizes (g/s)

\item {} 
\sphinxstyleliteralstrong{\sphinxupquote{loc}} \textendash{} an array of site ids

\item {} 
\sphinxstyleliteralstrong{\sphinxupquote{start}} \textendash{} an array of emission start times (days)

\item {} 
\sphinxstyleliteralstrong{\sphinxupquote{stop}} \textendash{} an array of emission stop times (days)

\item {} 
\sphinxstyleliteralstrong{\sphinxupquote{emtype}} \textendash{} an array of emission types (reparable or no)

\item {} 
\sphinxstyleliteralstrong{\sphinxupquote{em\_id}} \textendash{} an array of emission IDs

\end{itemize}

\item[{Returns}] \leavevmode
a FEAST Time object

\item[{Returns}] \leavevmode
a FEAST GasField object

\end{description}\end{quote}

\end{fulllineitems}



\subsection{load\_gas\_comp\_file}
\label{\detokenize{index:load-gas-comp-file}}\index{load\_gas\_comp\_file() (in module feast.MEET\_1\_importer)@\spxentry{load\_gas\_comp\_file()}\spxextra{in module feast.MEET\_1\_importer}}

\begin{fulllineitems}
\phantomsection\label{\detokenize{index:feast.MEET_1_importer.load_gas_comp_file}}\pysiglinewithargsret{\sphinxbfcode{\sphinxupquote{load\_gas\_comp\_file}}}{\emph{path\_to\_gas\_comp}}{}
Loads a gas\_comp.csv file generated by MEET 1.0
\begin{quote}\begin{description}
\item[{Parameters}] \leavevmode
\sphinxstyleliteralstrong{\sphinxupquote{path\_to\_gas\_comp}} \textendash{} a path (str) to a gas\_comp file

\item[{Returns}] \leavevmode
a DataFrame of the gas\_comp file

\item[{Returns}] \leavevmode
the time resolution of meet simulation (seconds)

\end{description}\end{quote}

\end{fulllineitems}



\section{ResultsProcessing}
\label{\detokenize{index:resultsprocessing}}

\subsection{plotting\_functions}
\label{\detokenize{index:module-feast.ResultsProcessing.plotting_functions}}\label{\detokenize{index:plotting-functions}}\index{feast.ResultsProcessing.plotting\_functions (module)@\spxentry{feast.ResultsProcessing.plotting\_functions}\spxextra{module}}

\subsubsection{abatement\_cost\_plotter}
\label{\detokenize{index:abatement-cost-plotter}}\index{abatement\_cost\_plotter() (in module feast.ResultsProcessing.plotting\_functions)@\spxentry{abatement\_cost\_plotter()}\spxextra{in module feast.ResultsProcessing.plotting\_functions}}

\begin{fulllineitems}
\phantomsection\label{\detokenize{index:feast.ResultsProcessing.plotting_functions.abatement_cost_plotter}}\pysiglinewithargsret{\sphinxbfcode{\sphinxupquote{abatement\_cost\_plotter}}}{\emph{directory}, \emph{gwp=34}}{}
Generates a box plot of the cost of abatement
gwp defaults to 34, which is the value provided in the IPCC 5th assessment report including climate\sphinxhyphen{}carbon feedbacks
(see Table 8.7, page 714 in Chapter 8 of Climate Change 2013: The Physical Science Basis.)
\begin{quote}\begin{description}
\item[{Parameters}] \leavevmode\begin{itemize}
\item {} 
\sphinxstyleliteralstrong{\sphinxupquote{directory}} \textendash{} A directory containing one or more realizations of a scenario

\item {} 
\sphinxstyleliteralstrong{\sphinxupquote{gwp}} \textendash{} global warming potential of methane

\end{itemize}

\item[{Returns}] \leavevmode


\end{description}\end{quote}

\end{fulllineitems}



\subsubsection{plot\_fixer}
\label{\detokenize{index:plot-fixer}}\index{plot\_fixer() (in module feast.ResultsProcessing.plotting\_functions)@\spxentry{plot\_fixer()}\spxextra{in module feast.ResultsProcessing.plotting\_functions}}

\begin{fulllineitems}
\phantomsection\label{\detokenize{index:feast.ResultsProcessing.plotting_functions.plot_fixer}}\pysiglinewithargsret{\sphinxbfcode{\sphinxupquote{plot\_fixer}}}{\emph{fig=None}, \emph{ax=None}, \emph{fsize=18}, \emph{color=(0}, \emph{0}, \emph{0)}, \emph{tight\_layout=True}, \emph{line\_width=4}, \emph{fontweight=\textquotesingle{}bold\textquotesingle{}}}{}
\end{fulllineitems}



\subsubsection{summary\_plotter}
\label{\detokenize{index:summary-plotter}}\index{summary\_plotter() (in module feast.ResultsProcessing.plotting\_functions)@\spxentry{summary\_plotter()}\spxextra{in module feast.ResultsProcessing.plotting\_functions}}

\begin{fulllineitems}
\phantomsection\label{\detokenize{index:feast.ResultsProcessing.plotting_functions.summary_plotter}}\pysiglinewithargsret{\sphinxbfcode{\sphinxupquote{summary\_plotter}}}{\emph{directory}, \emph{n\_wells=None}, \emph{ylabel=None}}{}
The NPV for each realization stored in ‘directory is calculated and displayed in a stacked bar chart. Each component
of the NPV is displayed separately in the chart.
\begin{quote}\begin{description}
\item[{Parameters}] \leavevmode\begin{itemize}
\item {} 
\sphinxstyleliteralstrong{\sphinxupquote{directory}} \textendash{} path to a directory containing results files

\item {} 
\sphinxstyleliteralstrong{\sphinxupquote{n\_wells}} \textendash{} if set to a number, then the NPV will be reported on a per well basis

\item {} 
\sphinxstyleliteralstrong{\sphinxupquote{ylabel}} \textendash{} yaxis label

\end{itemize}

\end{description}\end{quote}

\end{fulllineitems}



\subsubsection{time\_series}
\label{\detokenize{index:time-series}}\index{time\_series() (in module feast.ResultsProcessing.plotting\_functions)@\spxentry{time\_series()}\spxextra{in module feast.ResultsProcessing.plotting\_functions}}

\begin{fulllineitems}
\phantomsection\label{\detokenize{index:feast.ResultsProcessing.plotting_functions.time_series}}\pysiglinewithargsret{\sphinxbfcode{\sphinxupquote{time\_series}}}{\emph{results\_file}, \emph{line\_width=6}}{}
Display a time series of emissions from each detection method in a results file
\begin{quote}\begin{description}
\item[{Parameters}] \leavevmode\begin{itemize}
\item {} 
\sphinxstyleliteralstrong{\sphinxupquote{results\_file}} \textendash{} path to a results file

\item {} 
\sphinxstyleliteralstrong{\sphinxupquote{line\_width}} \textendash{} width at which to plot lines

\end{itemize}

\end{description}\end{quote}

\end{fulllineitems}



\subsection{results\_analysis\_functions}
\label{\detokenize{index:module-feast.ResultsProcessing.results_analysis_functions}}\label{\detokenize{index:results-analysis-functions}}\index{feast.ResultsProcessing.results\_analysis\_functions (module)@\spxentry{feast.ResultsProcessing.results\_analysis\_functions}\spxextra{module}}

\subsubsection{npv\_calculator}
\label{\detokenize{index:npv-calculator}}\index{npv\_calculator() (in module feast.ResultsProcessing.results\_analysis\_functions)@\spxentry{npv\_calculator()}\spxextra{in module feast.ResultsProcessing.results\_analysis\_functions}}

\begin{fulllineitems}
\phantomsection\label{\detokenize{index:feast.ResultsProcessing.results_analysis_functions.npv_calculator}}\pysiglinewithargsret{\sphinxbfcode{\sphinxupquote{npv\_calculator}}}{\emph{filepath}, \emph{discount\_rate}, \emph{gas\_price}}{}
Calculates the net present value (NPV) of each LDAR program in the results file
\begin{quote}\begin{description}
\item[{Parameters}] \leavevmode\begin{itemize}
\item {} 
\sphinxstyleliteralstrong{\sphinxupquote{filepath}} \textendash{} path to a results file

\item {} 
\sphinxstyleliteralstrong{\sphinxupquote{discount\_rate}} \textendash{} The discount rate of future cash flows (should be between 0 and 1)

\item {} 
\sphinxstyleliteralstrong{\sphinxupquote{gas\_price}} \textendash{} The value to assign to mitigated gas losses (\$/gram)

\end{itemize}

\item[{Returns}] \leavevmode
null\_npv          NPV of each LDAR program compared to a scenario with only the Null LDAR program {[}k\$/well{]}

\end{description}\end{quote}

\end{fulllineitems}



\subsubsection{results\_analysis}
\label{\detokenize{index:results-analysis}}\index{results\_analysis() (in module feast.ResultsProcessing.results\_analysis\_functions)@\spxentry{results\_analysis()}\spxextra{in module feast.ResultsProcessing.results\_analysis\_functions}}

\begin{fulllineitems}
\phantomsection\label{\detokenize{index:feast.ResultsProcessing.results_analysis_functions.results_analysis}}\pysiglinewithargsret{\sphinxbfcode{\sphinxupquote{results\_analysis}}}{\emph{directory}, \emph{discount\_rate}, \emph{gas\_price}}{}
Process many realizations of a single scenario stored in a directory
\begin{quote}\begin{description}
\item[{Parameters}] \leavevmode
\sphinxstyleliteralstrong{\sphinxupquote{directory}} \textendash{} A directory of results files all generated under the same scenario

\item[{Returns}] \leavevmode
null\_npv          array of null\sphinxhyphen{}NPV of each LDAR program in each realization {[}k\$/well{]}
emissions\_timeseries  Array of emissions in each LDAR program in each realization at each time step
costs                 Array of costs associated with each LDAR program (no discounting, all costs summed)
techs           list of detection program names

\end{description}\end{quote}

\end{fulllineitems}



\renewcommand{\indexname}{Python Module Index}
\begin{sphinxtheindex}
\let\bigletter\sphinxstyleindexlettergroup
\bigletter{f}
\item\relax\sphinxstyleindexentry{feast.DetectionModules.abstract\_detection\_method}\sphinxstyleindexpageref{index:\detokenize{module-feast.DetectionModules.abstract_detection_method}}
\item\relax\sphinxstyleindexentry{feast.DetectionModules.comp\_survey}\sphinxstyleindexpageref{index:\detokenize{module-feast.DetectionModules.comp_survey}}
\item\relax\sphinxstyleindexentry{feast.DetectionModules.ldar\_program}\sphinxstyleindexpageref{index:\detokenize{module-feast.DetectionModules.ldar_program}}
\item\relax\sphinxstyleindexentry{feast.DetectionModules.repair}\sphinxstyleindexpageref{index:\detokenize{module-feast.DetectionModules.repair}}
\item\relax\sphinxstyleindexentry{feast.DetectionModules.site\_monitor}\sphinxstyleindexpageref{index:\detokenize{module-feast.DetectionModules.site_monitor}}
\item\relax\sphinxstyleindexentry{feast.DetectionModules.site\_survey}\sphinxstyleindexpageref{index:\detokenize{module-feast.DetectionModules.site_survey}}
\item\relax\sphinxstyleindexentry{feast.EmissionSimModules.emission\_class\_functions}\sphinxstyleindexpageref{index:\detokenize{module-feast.EmissionSimModules.emission_class_functions}}
\item\relax\sphinxstyleindexentry{feast.EmissionSimModules.infrastructure\_classes}\sphinxstyleindexpageref{index:\detokenize{module-feast.EmissionSimModules.infrastructure_classes}}
\item\relax\sphinxstyleindexentry{feast.EmissionSimModules.result\_classes}\sphinxstyleindexpageref{index:\detokenize{module-feast.EmissionSimModules.result_classes}}
\item\relax\sphinxstyleindexentry{feast.EmissionSimModules.simulation\_classes}\sphinxstyleindexpageref{index:\detokenize{module-feast.EmissionSimModules.simulation_classes}}
\item\relax\sphinxstyleindexentry{feast.input\_data\_classes}\sphinxstyleindexpageref{index:\detokenize{module-feast.input_data_classes}}
\item\relax\sphinxstyleindexentry{feast.MEET\_1\_importer}\sphinxstyleindexpageref{index:\detokenize{module-feast.MEET_1_importer}}
\item\relax\sphinxstyleindexentry{feast.ResultsProcessing.plotting\_functions}\sphinxstyleindexpageref{index:\detokenize{module-feast.ResultsProcessing.plotting_functions}}
\item\relax\sphinxstyleindexentry{feast.ResultsProcessing.results\_analysis\_functions}\sphinxstyleindexpageref{index:\detokenize{module-feast.ResultsProcessing.results_analysis_functions}}
\end{sphinxtheindex}

\renewcommand{\indexname}{Index}
\printindex
\end{document}